
\chapter{Evaluation of other OSS Verilog Synthesis Tools}
\label{chapter:sota}

In this appendix\footnote{This appendix is an updated version of an
unpublished student research paper. \cite{VerilogFossEval}}
the existing FOSS Verilog synthesis tools\footnote{To the
author's best knowledge, all relevant tools that existed at the time of this
writing are included. But as there is no formal channel through which such
tools are published it is hard to give any guarantees in that matter.} are
evaluated. Extremely limited or application specific tools (e.g.~pure Verilog
Netlist parsers) as well as Verilog simulators are not included. These existing
solutions are tested using a set of representative Verilog code snippets. It is
shown that no existing FOSS tool implements even close to a sufficient subset
of Verilog to be usable as synthesis tool for a wide range existing Verilog code.

The packages evaluated are:

\begin{itemize}
\item Icarus Verilog \citeweblink{Icarus}\footnote{Icarus Verilog is mainly a simulation
tool but also supported synthesis up to version 0.8. Therefore version 0.8.7 is used
for this evaluation.)}
\item Verilog-to-Routing (VTR) / Odin-II \cite{vtr2012}\cite{Odin}\citeweblink{VTR}
\item HDL Analyzer and Netlist Architect (HANA) \citeweblink{HANA}
\item Verilog front-end to VIS (vl2mv) \cite{Cheng93vl2mv:a}\citeweblink{VIS}
\end{itemize}

In each of the following sections Verilog modules that test a certain Verilog
language feature are presented and the support for these features is tested in all
the tools mentioned above. It is evaluated whether the tools under test
successfully generate netlists for the Verilog input and whether these netlists
match the simulation behavior of the designs using testbenches.

All test cases are verified to be synthesizeable using Xilinx XST from the Xilinx
WebPACK \citeweblink{XilinxWebPACK} suite.

Trivial features such as support for simple structural Verilog are not explicitly tested.

Vl2mv and Odin-II generate output in the BLIF (Berkeley Logic Interchange
Format) and BLIF-MV (an extended version of BLIF) formats respectively.
ABC \citeweblink{ABC} is used to convert this output to Verilog for verification
using testbenches.

Icarus Verilog generates EDIF (Electronic Design Interchange Format) output
utilizing LPM (Library of Parameterized Modules) cells. The EDIF files are
converted to Verilog using edif2ngd and netgen from Xilinx WebPACK. A
hand-written implementation of the LPM cells utilized by the generated netlists
is used for verification.

Following these functional tests, a quick analysis of the extensibility of the tools
under test is provided in a separate section.

The last section of this chapter finally concludes these series of evaluations
with a summary of the results.

\begin{figure}[t!]
	\begin{minipage}{7.7cm}
		\lstinputlisting[numbers=left,frame=single,language=Verilog]{CHAPTER_StateOfTheArt/always01_pub.v}
	\end{minipage}
	\hfill
	\begin{minipage}{7.7cm}
		\lstinputlisting[frame=single,language=Verilog]{CHAPTER_StateOfTheArt/always02_pub.v}
	\end{minipage}
	\caption{1st and 2nd Verilog always examples}
	\label{fig:StateOfTheArt_always12}
\end{figure}

\begin{figure}[!]
	\lstinputlisting[numbers=left,frame=single,language=Verilog]{CHAPTER_StateOfTheArt/always03.v}
	\caption{3rd Verilog always example}
	\label{fig:StateOfTheArt_always3}
\end{figure}

\section{Always blocks and blocking vs.~nonblocking assignments}
\label{sec:blocking_nonblocking}

The ``always''-block is one of the most fundamental non-trivial Verilog
language features.  It can be used to model a combinatorial path (with optional
registers on the outputs) in a way that mimics a regular programming language.

Within an always block, if- and case-statements can be used to model multiplexers.
Blocking assignments ($=$) and nonblocking assignments ($<=$) are used to populate the
leaf-nodes of these multiplexer trees. Unassigned leaf-nodes default to feedback
paths that cause the output register to hold the previous value. More advanced
synthesis tools often convert these feedback paths to register enable signals or
even generate circuits with clock gating.

Registers assigned with nonblocking assignments ($<=$) behave differently from
variables in regular programming languages: In a simulation they are not
updated immediately after being assigned.  Instead the right-hand sides are
evaluated and the results stored in temporary memory locations. After all
pending updates have been prepared in this way they are executed, thus yielding
semi-parallel execution of all nonblocking assignments.

For synthesis this means that every occurrence of that register in an expression
addresses the output port of the corresponding register regardless of the question whether the register
has been assigned a new value in an earlier command in the same always block.
Therefore with nonblocking assignments the order of the assignments has no effect
on the resulting circuit as long as the left-hand sides of the assignments are
unique.

The three example codes in Fig.~\ref{fig:StateOfTheArt_always12} and
Fig.~\ref{fig:StateOfTheArt_always3} use all these features and can thus be used
to test the synthesis tools capabilities to synthesize always blocks correctly.

The first example is only using the most fundamental Verilog features. All
tools under test were able to successfully synthesize this design.

\begin{figure}[b!]
	\lstinputlisting[numbers=left,frame=single,language=Verilog]{CHAPTER_StateOfTheArt/arrays01.v}
	\caption{Verilog array example}
	\label{fig:StateOfTheArt_arrays}
\end{figure}

The 2nd example is functionally identical to the 1st one but is using an
if-statement inside the always block. Odin-II fails to synthesize it and
instead produces the following error message:

\begin{verbatim}
ERROR: (File: always02.v) (Line number: 13)
You've defined the driver "count~0" twice
\end{verbatim}

Vl2mv does not produce an error message but outputs an invalid synthesis result
that is not using the reset input at all.

Icarus Verilog also doesn't produce an error message but generates an invalid output
for this 2nd example. The code generated by Icarus Verilog only implements the reset
path for the count register, effectively setting the output to constant 0.

So of all tools under test only HANA was able to create correct synthesis results
for the 2nd example.

The 3rd example is using blocking and nonblocking assignments and many if statements.
Odin also fails to synthesize this example:

\begin{verbatim}
ERROR: (File: always03.v) (Line number: 8)
ODIN doesn't handle blocking statements in Sequential blocks
\end{verbatim}

HANA, Icarus Verilog and vl2mv create invalid synthesis results for the 3rd example.

So unfortunately none of the tools under test provide a complete and correct
implementation of blocking and nonblocking assignments.

\section{Arrays for memory modelling}

Verilog arrays are part of the synthesizeable subset of Verilog and are
commonly used to model addressable memory. The Verilog code in
Fig.~\ref{fig:StateOfTheArt_arrays} demonstrates this by implementing a single
port memory.

For this design HANA, vl2m and ODIN-II generate error messages indicating that
arrays are not supported.

\begin{figure}[t!]
	\lstinputlisting[numbers=left,frame=single,language=Verilog]{CHAPTER_StateOfTheArt/forgen01.v}
	\caption{Verilog for loop example}
	\label{fig:StateOfTheArt_for}
\end{figure}

Icarus Verilog produces an invalid output that is using the address only for
reads. Instead of using the address input for writes, the generated design
simply loads the data to all memory locations whenever the write-enable input
is active, effectively turning the design into a single 4-bit D-Flip-Flop with
enable input.

As all tools under test already fail this simple test, there is nothing to gain
by continuing tests on this aspect of Verilog synthesis such as synthesis of dual port
memories, correct handling of write collisions, and so forth.

\begin{figure}[t!]
	\lstinputlisting[numbers=left,frame=single,language=Verilog]{CHAPTER_StateOfTheArt/forgen02.v}
	\caption{Verilog generate example}
	\label{fig:StateOfTheArt_gen}
\end{figure}

\section{For-loops and generate blocks}

For-loops and generate blocks are more advanced Verilog features. These features
allow the circuit designer to add program code to her design that is evaluated
during synthesis to generate (parts of) the circuits description; something that
could only be done using a code generator otherwise.

For-loops are only allowed in synthesizeable Verilog if they can be completely
unrolled. Then they can be a powerful tool to generate array logic or static
lookup tables. The code in Fig.~\ref{fig:StateOfTheArt_for} generates a circuit that
tests a 5 bit value for being a prime number using a static lookup table.

Generate blocks can be used to model array logic in complex parametric designs. The
code in Fig.~\ref{fig:StateOfTheArt_gen} implements a ripple-carry adder with
parametric width from simple assign-statements and logic operations using a Verilog
generate block.

All tools under test failed to synthesize both test cases. HANA creates invalid
output in both cases. Icarus Verilog creates invalid output for the first
test and fails with an error for the second case. The other two tools fail with
error messages for both tests.

\section{Extensibility}

This section briefly discusses the extensibility of the tools under test and
their internal data- and control-flow.  As all tools under test already failed
to synthesize simple Verilog always-blocks correctly, not much resources have
been spent on evaluating the extensibility of these tools and therefore only a
very brief discussion of the topic is provided here.

HANA synthesizes for a built-in library of standard cells using two passes over
an AST representation of the Verilog input. This approach executes fast but
limits the extensibility as everything happens in only two comparable complex
AST walks and there is no universal intermediate representation that is flexible
enough to be used in arbitrary optimizations.

Odin-II and vl2m are both front ends to existing synthesis flows. As such they
only try to quickly convert the Verilog input into the internal representation
of their respective flows (BLIF). So extensibility is less of an issue here as
potential extensions would likely be implemented in other components of the
flow.

Icarus Verilog is clearly designed to be a simulation tool rather than a
synthesis tool. The synthesis part of Icarus Verilog is an ad-hoc add-on to
Icarus Verilog that aims at converting an internal representation that is meant
for generation of a virtual machine based simulation code to netlists.

\section{Summary and Outlook}

Table~\ref{tab:StateOfTheArt_sum} summarizes the tests performed. Clearly none
of the tools under test make a serious attempt at providing a feature-complete
implementation of Verilog. It can be argued that Odin-II performed best in the
test as it never generated incorrect code but instead produced error messages
indicating that unsupported Verilog features where used in the Verilog input.

In conclusion, to the best knowledge of the author, there is no FOSS Verilog
synthesis tool other than Yosys that is anywhere near feature completeness and
therefore there is no other candidate for a generic Verilog front end and/or
synthesis framework to be used as a basis for custom synthesis tools.

Yosys could also replace vl2m and/or Odin-II in their respective flows or
function as a pre-compiler that can translate full-featured Verilog code to the
simple subset of Verilog that is understood by vl2m and Odin-II.

Yosys is designed for extensibility. It can be used as-is to synthesize Verilog
code to netlists, but its main purpose is to be used as basis for custom tools.
Yosys is structured in a language dependent Verilog front end and language
independent synthesis code (which is in itself structured in independent
passes). This architecture will simplify implementing additional HDL front
ends and/or additional synthesis passes.

Chapter~\ref{chapter:eval} contains a more detailed evaluation of Yosys using real-world
designs that are far out of reach for any of the other tools discussed in this appendix.

\vskip2cm
\begin{table}[h]
	%             yosys       hana        vis         icarus      odin
	% always01    ok          ok          ok          ok          ok
	% always02    ok          ok          failed      failed      error
	% always03    ok          failed      failed      missing     error
	% arrays01    ok          error       error       failed      error
	% forgen01    ok          failed      error       failed      error
	% forgen02    ok          failed      error       error       error
	\def\ok{\ding{52}}
	\def\error{\ding{56}}
	\def\failed{$\skull$}
	\def\missing{$\skull$}
	\rowcolors{2}{gray!25}{white}
	% \centerline{
		\begin{tabular}{|l|cccc|c|}
		\centering
		\hline
		& \bf HANA & \bf VIS / vl2m & \bf Icarus Verilog & \bf Odin-II & \bf Yosys \\
		\hline
		\tt always01 &  \ok       &  \ok       &  \ok       &  \ok     & \ok \\
		\tt always02 &  \ok       &  \failed   &  \failed   &  \error  & \ok \\
		\tt always03 &  \failed   &  \failed   &  \missing  &  \error  & \ok \\
		\tt arrays01 &  \error    &  \error    &  \failed   &  \error  & \ok \\
		\tt forgen01 &  \failed   &  \error    &  \failed   &  \error  & \ok \\
		\tt forgen02 &  \failed   &  \error    &  \error    &  \error  & \ok \\
		\hline
		\end{tabular}
	% }
	\centerline{
		\ding{52} \dots passed \hskip2em
		\ding{56} \dots produced error \hskip2em
		$\skull$ \dots incorrect output
	}
	\caption{Summary of all test results}
	\label{tab:StateOfTheArt_sum}
\end{table}
